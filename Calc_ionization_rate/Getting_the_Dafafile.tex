\documentclass[UTF8]{article}
\usepackage{amsmath}
\usepackage{ctex}
\usepackage{geometry}
\usepackage{graphicx}
\usepackage{float}
\usepackage{array}
\usepackage{longtable}
\usepackage{multirow}
\usepackage{bigstrut}
\usepackage{mhchem}

\DeclareGraphicsExtensions{.eps,.ps,.jpg,.bmp}
\geometry{top=3.18cm,bottom=3.18cm}
\begin{document}
	\bibliographystyle{unsrt}
	\pagestyle{plain}
	\title{\textbf{The way to calculate the ionization rate and the KROME datafile}}
	\author{}
	\date{}
	\maketitle
	\renewcommand\arraystretch{1.3}
	Based on Latif+ (2015).\\
	$$\zeta_x^i=\zeta_p^i+\sum_{j=H,He}\frac{n_j}{n_i}\zeta_p^j\langle\phi^j\rangle$$
	where $n_j$ is the number density.
	$$\zeta_p^i=\frac{4\pi}{h}\int_{E_{min}}^{E_{max}}\frac{J(E)}{E}e^{-\tau(E)}\sigma^i(E)\text{d}E$$
	$$J(E)=J_{X,21}\left( \frac{E}{1\text{keV}}\right) ^{-1.5}\times10^{-21}\text{ erg cm}^{−2}\text{ s}^{-1}\text{ Hz}^{-1} \text{ sr}^{-1}$$
	where $J_{X,21}=1$
	$$\tau(E)=\sum_{i=H,He}N_i\sigma^i=\frac{\lambda_J}{2}\sum_{i=H,He}n_i\sigma^i$$
	Jeans length:
	$$\lambda_J=\sqrt{\frac{\pi kT}{G\rho \mu m_p}}$$
	$$\mu=\frac{1.00794n_H+4.0026022n_{He}}{n_H+n_{He}}$$
	$\sigma^i$ comes from Verner $\&$ Ferland (1996)\\
	$\langle\phi^j\rangle$ is much more complex. For $E>100$eV and H, He mixture
	$$\phi^H(E,x_e)=\left( \frac{E}{13.6\text{eV}}-1\right)0.3908(1-x_e^{0.4092})^{1.7592} $$
	$$\phi^He(E,x_e)=\left( \frac{E}{24.6\text{eV}}-1\right)0.0554(1-x_e^{0.4614})^{1.666} $$
	where $x_e$ is the electron fraction
	$$\langle\phi^i\rangle=\frac{\int J(E)\phi^i(E,x_e)\text{d}E}{\int J(E)\text{d}E}$$
	
	\noindent In the datafile that KROME offers:
	$$E_{min}=2\text{ keV}\qquad E_{max}=10\text{ keV}$$
	$$T=160\text{ K}$$
	$$J_{X,21}=1$$
	The datafile is a $30\time30$ table which shows log($N_H$), log($N_{H}$)-log($N_{He}$), $N$ is the column density. The size of the cloud is estimated by its Jeans length and the number density $n$
	$$n^i=\frac{N^i}{\lambda_J/2}$$
	To get the certain X-ray ionization rate, the KROME package first turns the number densities(H, He) into column densities. Notice that the Jeans length now does not only depend on H/He but other atoms. then it reads the datafile rateH.dat/rateHe.dat and does 2-Dimension linear interpolation to find the ionization rate. This is how these 'auto' rates in the network are calculated.
	
\end{document}
