\documentclass{aastex62}
\usepackage{amsmath}
\usepackage{natbib}
\usepackage{mhchem}
\bibliographystyle{apj}
\begin{document}
	\title{Chemical impacts of X-ray from an active supermassive black hole in our galaxy}
	\author{Chang Liu}
	\affiliation{Department of Astronomy, Peking University}
	\begin{abstract}
		This is an abstract...
	\end{abstract}
	\keywords{To be continued...}
	\section{Introduction}
	
	Supermassive black holes (SMBHs) are widely held in galaxies. The bright radio object Sagittarius A* (Sgr A*) located in the galactic centre 8 kpc away from earth is believed as the SMBH in our galaxy. It weighs 
	$4\times10^{6}M_\odot$ and is quite faint with surprisingly low luminosity $L\sim10^{33-35}\text{erg s}^{-1}$. This is no more 
	than $10^{-9}L_{\text{Edd}}$, where $L_{\mathrm{Edd}} \approx 1.3 \times 10^{38}\left(M_{\mathrm{BH}} / M_{\odot}\right) \text{erg} 
	\mathrm{s}^{-1}\sim 5\times10^{44} \text{erg} \mathrm{s}^{-1}$ is the Eddington luminosity \citep{Sabha2010}, showing that Sgr A* is on quiescent state. 
	
	More recent researches challenge the conventional understanding of Observations have shown evidence that Sgr A* went through an active phase 6 million years ago \citep{Nicastro2016}. Activation of Sgr A* triggers radiation of hard X-ray photons. For a hydrogen atom, the photo-ionization cross section $\sigma_p^{\ce{H}}$ is proportional to $(h\nu)^{-3}$ \citep{Brown1970}, where $\nu$ is the frequency of the photon, indicating tiny cross sections with high energetic photons. Hard X-ray photons can therefore transmit much farther than optical and UV photons in galactic disk. \cite{Amaro-Seoane2014} calculated the X-ray irradiation from Sgr A* due to accretion of gas (like an AGN) and the tidal disruption, Sgr A* could precipitate on Earth a hard X-ray (i.e. $h\nu > 2$ keV) flux comparable to that from the current quiescent sun, while UV and soft X-ray photons suffer from heavy extinction and are not significant 8 kpc away from galactic centre. These energetic photons may leave significant physical and chemical records in molecular gas around galactic centre\citep{Krolik1983, Neufeld1994, Aalto2014} and planetary atmospheres \citep{Loeb2018,Wis2019}.
	In our case, we focus on the chemical evolution in dense molecular clouds under the X-ray radiation of an active Sgr A*. 
	\section{Methods}
	\subsection{X-ray flux in different distances from Sgr A*}
	\subsection{X-ray chemistry}
	\subsubsection{Primary ionization}
	Hard X-ray photons significantly influence the ionization fraction of the neutral molecular cloud and thus influence its thermal and chemical evolution. When a X-ray photon comes across an atom
	\subsubsection{Secondary ionization}
	\subsubsection{Ionization of heavy elements and molecules}
	\subsection{Cosmic-ray chemistry}
	\subsection{Chemical networks}
	
	\section{Models}
	\subsection{X-ray models}
	\subsection{Molecular cloud models}
	Pseudo-time-dependent approach
	KROME package\footnote{\url{http://kromepackage.org/}} \citep{Grassi2014}
	
	\section{Results}
	
	\section{Discussions}
	\bibliography{ack.bib}

	
\end{document}
